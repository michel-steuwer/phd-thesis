% Chapter 8: Conclusion

\chapter{Conclusion}

\label{ch:eighth} % For referencing the chapter elsewhere, use \autoref{ch:name} 

\section{Conclusion}

%\from{HIPS begin}
%\subsection{HIPS}
%We developed and implemented SkelCL -- an OpenCL-based skeleton library for high-level GPU programming, based on an abstract data type and algorithmic skeletons.
%Currently, it provides a vector data type and four basic skeletons (\texttt{Map}, \texttt{Zip}, \texttt{Reduce}, \texttt{Scan}).
%SkelCL shields the user from the low-level details of GPU programming.
%Data transfer and synchronization are performed implicitly.
%
%Our application examples show that SkelCL provides competitive performance and scalability on real-world applications as compared with CUDA and OpenCL.
%While SkelCL adds a minor performance overhead, it significantly reduces the programming effort, since much of the boilerplate code required in CUDA or OpenCL is replaced by shorter and more intuitive high-level constructs.
%\from{HIPS end}
%
%\from{ASHES begin}
%\subsection{ASHES}
%In this paper, we presented SkelCL -- a high-level multi-GPU programming library.
%The novel contributions of SkelCL are two-fold:
%1)~using the vector data type and the built-in mechanism of data distributions, it considerably simplifies memory management in multi-GPU programs;
%2)~the high-level algorithmic skeletons are used for programming multi-GPU systems, resulting in shorter, better structured programs as compared to OpenCL and CUDA.
%Moreover, SkelCL extends the flexibility of skeletons with its additional arguments feature, as demonstrated on a real-world, medical imaging application.
%
%Our case study showed that SkelCL significantly reduces programming effort in terms of lines of code, and greatly improves program structure and maintainability, while it causes less than 5\% performance decrease as compared to the low-level OpenCL-implementation.
%We obtained similar results about the programming effort and performance for the Mandelbrot benchmark application~\cite{SteuwerKeGo2011}.
%\from{ASHES end}
%
%\from{Paraphrase begin}
%\subsection{Paraphrase}
%In this paper we showed how the SkelCL library can be extended for developing applications on two-dimensional data.
%We used an image processing application as a case study.
%Using SkelCL, such applications can easily benefit from the performance of GPUs.
%Application developers do not have to be GPU computing experts to achieve good performance, since SkelCL's skeletons exploit the GPU memory hierarchy transparently for the user.
%The two-dimensional data type significantly simplifies memory management.
%The SkelCL library is available as open-source software at \texttt{http://skelcl.uni-muenster.de}.
%\from{Paraphrase end}
%
%\from{ICCS begin}
%\subsection{ICCS}
%This paper presented the SkelCL high-level programming model for multi-GPU systems and its implementation as a library.
%We focused on programming methodology and, therefore, deliberately restricted ourselves to a single sample real-world application as motivation example and benchmark for experiments.
%Additional application examples can be found on the SkelCL website \url{http://skelcl.uni-muenster.de}, including LU decomposition, computation of the Mandelbrot set, matrix multiplication, Jacobi stencil computations, B+ tree traversal, the Mersenne Twister, etc. SkelCL is freely available as open source software.
%
%The SkelCL programming model significantly raises the level of abstraction: it combines parallel patterns to express computations, parallel container data types for simplified memory management and a data (re)distribution mechanism to improve scalability in systems with multiple GPUs.
%Our SkelCL library significantly reduces the amount of source code necessary to implement the sample imaging application (by 50\%) and frees the application developer from low-level memory management and other tedious programming tasks.
%The performance experiments show that SkelCL introduces a moderate overhead of less than 5\% as compared to the arguably more complicated and error-prone OpenCL implementation.
%\from{ICCS end}
%
%\from{PaCT begin}
%\subsection{PaCT}
%This paper presents the SkelCL high-level programming model for multi-GPU systems and its implementation as a library.  
%The SkelCL programming model significantly raises the level of abstraction: it combines parallel patterns to express computations, parallel container data types for simplified memory management and a data (re)distribution mechanism to improve scalability in systems with multiple GPUs.
%The SkelCL library is available as open source software from \url{http://skelcl.uni-muenster.de}.
%\from{PaCT end}
%
%\from{HiStencils begin}
%\subsection{HiStencils}
%In the paper, we describe how stencil computations are programmed in our SkelCL approach that combines high level of programming abstraction with competitive performance on multi-GPU systems.
%We introduce two SkelCL skeletons for stencil computations -- MapOverlap and Stencil -- and we discuss their efficient parallel implementation, and report experimental results.
%We demonstrate that when executing a single stencil shape once, the MapOverlap skeleton should be used;
%in all other cases, the Stencil skeleton is the better choice regarding both user comfort and performance.
%Both skeletons meet SkelCL's requirements of offering high levels of programming abstraction together with a competitive performance on multiple devices, and yield much shorter and simpler codes than when using OpenCL.
%\from{HiStencils end}
%
%\from{PACT begin}
%\subsection{PACT}
%In this paper, we have presented a set of rewrite rules that automatically transform high-level algorithmic expressions into compositions of low-level hardware patterns. 
%Our rules allow us to express existing optimization strategies, found in hand-optimized code, as well as discover new ones with superior performance.
%As we automatically explore the space of possible implementations with our rules, as opposed to hard-coded optimization strategies, we are able to provide performance portability.
%
%We have demonstrated that our approach achieves performance on par with highly tuned platform specific BLAS libraries.
%For benchmarks such as matrix vector multiplication we even reach speedup of up to 4.5$\times$.
%We also show that our technique achieves portable performance for more complex applications such as the BlackScholes benchmark or for molecular dynamics simulation.
%\from{PACT end}

