% Chapter 8: Conclusion

\chapter{Conclusion and Comparison to Related Work}

\label{ch:eighth} % For referencing the chapter elsewhere, use \autoref{ch:name}

In this chapter we compare the approaches presented in this thesis to related work and draw final conclusions.

\section{Related Work}
Here we will discuss related projects which also aim to simplify parallel programming in general or of GPU systems in particular.
We also include projects aiming for performance portability, as our approach does.
We will start by looking at algorithmic skeleton libraries in general and then focus on more recent projects targeting \GPU systems, like \SkelCL does.
Next, we will cover other structured parallel programming approaches, including the famous \emph{MapReduce} framework.
We will then discuss the broad range of \GPU programming approaches proposed in recent years, before looking at domain specific approaches, including projects particular focus on stencil computations.
We will end with a discussion of related projects using rewrite rules for program optimizations.

For all projects discuss we will make clear how they relate to our work.
% We will see, that no other project exist which combines a practical high-level programming approach


\subsection{Algorithmic Skeleton Libraries}
Numerous algorithmic skeleton libraries have been proposed since the introduction of algorithmic skeletons in the late 1980s~\cite{Cole1991}.
A good and extensive overview reflecting the state of the art at the time when the work on this thesis was started in 2010 can be found in~\cite{Gonzalez-VelezL10}.
We will discuss here some representative examples of algorithmic skeleton libraries targeting different types of computer architectures.

Prominent algorithmic skeleton libraries targeting distributed systems are \emph{Muesli}~\cite{Kuchen02} and \emph{eSkel}~\cite{Cole04} which are both implemented using MPI~\cite{MPI}.
There has also been work especially dedicated towards grids~\cite{AltG03a, Alt2007} leading to the development of the \emph{Higher Order Components (HOC)}~\cite{DunnweberG04,DuennweberG09} which are implemented in Java.

Skeleton libraries for multicore \CPUs include \emph{Skandium}~\cite{LeytonP10} which uses Java threads, \emph{FastFlow}~\cite{AldinucciDaKiTo2011,AldinucciDKMT11} which is implemented in \Cpp and has recently be extended towards distributed systems as well~\cite{AldinucciCDKT12}, and an extended version of \emph{Muesli}~\cite{CiechanowiczK10} which uses OpenMP~\cite{OpenMP}.

\bigskip

Of particular relevance for our comparison are the following recent skeleton libraries targeting \GPU systems.

\bigskip

\emph{Muesli}~\cite{ErnstingK12} and \emph{FastFlow}~\cite{BuonoDLT13,AldinucciSDTP12} have been extended for \GPU systems using \CUDA.
Both libraries implemented support for execution of their data-parallel skeletons on \GPU hardware, but not for their task-parallel skeletons.
In Muesli data-parallel skeletons can be nested in task-parallel skeletons, but not the other way around.
This type of nesting is also supported when the data-parallel skeleton is executed on a \GPU.
The data management between \CPU and \GPU is performed implicitly and automatically as it is the case for \SkelCL, but different to our implementation data is transfered back to the \CPU after each skeleton execution on the \GPU.
This makes the integration with the existing infrastructure in Muesli and FastFlow easier but obviously limits performance when multiple skeletons are executed on the \GPU.

\bigskip

\emph{SkePU}~\cite{EnmyrenKe10,DastgeerEnKe2011,DastgeerKe14} is a skeleton library implemented in \Cpp and specifically targeted towards \GPU systems, similar to \SkelCL.
Both approaches have been developed independently but implement very similar concepts and even a similar set of data-parallel algorithmic skeletons.
Nevertheless, both projects have been implemented with emphasis on different areas and are implemented in different ways.
SkePU implements multiple backends for targeting different hardware devices.
Currently, there exists an OpenMP backend for multicore \CPUs, OpenCL and CUDA backends for \GPUs, and separate backends written in OpenCL and CUDA for multi-\GPU execution.

The approach of developing multiple backends is contradictory to the idea of code and performance portability advocated in this thesis.
\SkelCL uses only a single \OpenCL backend which will be combined in the future with our novel compiler technique to optimize code for different platforms.

Recently SkePU has implemented a similar scheme as SkelCL for managing data transfers~\cite{DastgeerKe14}, by using a similar lazy copying strategy then SkelCL does since its first implementation.
SkePU now also supports to automatically overlap data transfer with computations, which is currently not supported in \SkelCL.

A version of SkePU exists, which is integrated in the StarPU runtime system~\cite{AugonnetTNW09} and allows for hybrid \CPU and \GPU execution with an dynamic load balancing system provided by StarPU.
Furthermore, SkePU allows to specify \emph{plans} which determine the backend to be used for a particular data size of the problem, \eg, the OpenMP backend for small data size, but the CUDA multi-\GPU backend for larger data sizes.
While SkelCL also fully support the execution on multicore \CPUs, single \GPUs, and  multi-\GPU systems, there is currently no comparable mechanism to determine which hardware should be used for different data sizes.

\SkelCL introduces data distributions to give users control over the execution in multi-\GPU systems.
SkePU does not offer such a feature and always splits the data across \GPUs, therefore, complicated multi-\GPU applications like the LM OSEM presented and evaluated in \autoref{chapter:skelcl-evaluation} are not easily supported by SkePU.

\bigskip

\emph{JPAI}~\cite{FumeroStDu2014} is a recent skeleton library for seamlessly programming \GPU systems from Java.
JPAI offers an object oriented API which makes use of the new Java lambda expressions.
At runtime before execution on the \GPU the customizing functions of the skeletons are compiled to \OpenCL using the Graal~\cite{DuboscqStWuSiWiMo2013} compiler and virtual machine.
There is currently no support for multi-\GPU systems, as there is in \SkelCL.


\subsection[Other Structured Parallel Programming\\ Approaches]{Other Structured Parallel Programming Approaches}
There are other projects advocating structured parallel programming, even though they identify themselves not necessary as algorithmic skeleton libraries.

\bigskip

\emph{Delite}~\cite{ChafiSBLAO11,LeeBSCROO11,BrownSLRCOO11} is a framework for building domain specific languages which automatically exploit parallelism.
To achieve this Delite offers a set of basic parallel operators, very similar to algorithmic skeletons, which can be used as fundamental building blocks by the designer of a domain specific language.
The domain specific language is compiled by Delite where \Cpp code is generated for multicore \CPUs and \CUDA code for \GPUs.
For performing the compilation Delite uses a project called \emph{Lightweight Modular Staging  (LMS)}~\cite{RompfO12} which is a runtime code generation approach implemented as a library in the Scala programming language~\cite{Odersky06,OderskyR14}.
LMS exploits the rich type system of Scala to give fine grained control over which expressions should be evaluated in Scala at compile time and for which expressions code should be generated.
Neither Delite nor LMS address the performance portability issue we identified in this thesis and address with our novel compilation technique.

\bigskip

% Map-Reduce (multiple)
\emph{MapReduce}~\cite{DeanG04} is a programming model advocated to simplify the programming of applications processing large amounts of data.
Often these applications run inside of data centers, the cloud, or other possibly large scale distributed systems.
Computations are divided into two steps called map and reduce.
In the map step a user provided function is applied to each data item, usually represented as a key-value pair, in a possibly large collection of data.
After the map step all matching values with the same key are grouped together and then in the reduce step all values for a given key can be aggregated by a second user provided function.
These concepts are closely related to algorithmic skeletons, even though a slightly different terminology is used and only computations fitting this one computational pattern can be expressed.
L{\"a}mmel discusses extensively MapReduce from a functional programming persecutive and explores its foundations in skeleton programming~\cite{Laemmel2007}.
The generic algorithmic skeleton approach discussed in this thesis allows for implementing a broad range of applications, as shown in \autoref{chapter:skelcl-evaluation} and \autoref{chapter:codeGeneration-evaluation}, and is not fixed to one particular application domain as MapReduce is.

Since its introduction MapReduce has found widespread use and several projects providing implementations on different architectures have been presented.
The most prominent example is \emph{Hadoop}~\cite{Hadoop} an open source implementation in Java targeting cluster systems.
Other work has targeted single- and multi-\GPU systems~\cite{StuartO11,FangHLG11}.

\bigskip

\emph{Threading Building Blocks (TBB)}~\cite{Reinders2007} is a software library developed by Intel.
TBB offers parallel patterns, like \code{parallel\_for} or \code{parallel\_reduce}, as well as concurrent data structures, including \code{concurrent\_queue}, \code{concurrent\_vector}, and \code{concurrent\_hash\_map}.
TBB can be used for programming of Intels multicore \CPUs and has recently been enabled for the Xeon Phi accelerators as well.
In a separate book~\cite{McCoolRoRe2012} three authors from Intel discuss how TBB, among other technologies advocated by Intel, can be used for structured parallel programming.
The \emph{\Cpp Extensions for Parallelism}~\cite{CppParallelism} is a set of proposed extensions to the \Cpp standard adding parallel algorithms, similar to algorithmic skeletons, to \Cpp.
TBB as well as existing skeleton libraries, including \SkelCL, could implemented the proposed specification in the future and, thus, conform to a unified and standardized programming interface.

\SkelCL currently is not optimized for multicore \CPUs as TBB is.
When combined with our novel compilation approach \SkelCL will aim for generating highly efficient code for multicore \CPUs in the future as well.

\subsection{Related \GPU programming approaches}
We already discussed some closely related approaches which can be used for program \GPU systems, including \emph{SkePU}, \emph{Muesli}, \emph{FastFlow}, and \emph{JPAI}.
Here we are going to discuss additional approach which are especially targeted towards \GPU programming.

\bigskip

\emph{Thrust}~\cite{BellHo2011} and \emph{Bolt}~\cite{Thrust} are \Cpp libraries developed by Nvidia and AMD respectively for simplify the programming of their \GPUs.
Both libraries offer a similar set of parallel patterns and interface to TBB and \SkelCL.
Thrust is implemented using \CUDA and Bolt uses \OpenCL, similar as \SkelCL does.
For data management both libraries offer separate data structures for the \CPU and \GPU.
The programmer has explicit control, but also the additional burden, to move data between \CPU and \GPU.
This is different to \SkelCL's implementation where a unified data structure is provided which automatically and lazily manages the data transfers in the system.

Currently, neither Thrust nor Bolt support multi-\GPU execution.
In \SkelCL multi-\GPU support is a key feature built into the programming model with the support of data distributions giving programmers control over how the \GPUs in the system should be used.

\bigskip

% Accelerate (or => other related GPU approaches)
\emph{Obsidian}~\cite{SvenssonSC08,SvenssonCS10} is a project for performing \GPU programming form the functional programming language Haskell~\cite{HudakPWBFFGHHJKNPP92}.
\GPU code is generated from expressions written in a domain specific language embedded in Haskell.
Obsidian offers low-level primitives to the programmer for achieving competitive performance with manually written \CUDA code.

\bigskip

\emph{Accelerate}~\cite{ChakravartyKLMG11,McDonellCKL13} is also a domain specific languages for data-parallel programming embedded in Haskell.
Accelerate makes use of Obsidian for generating \GPU code and offers a higher level interface to the programmer than Obsidian does.
Parallel combinators, like map, fold, or scan, are used to express programs at the algorithmic level.
This is very similar to how functional programmers usually express in ordinary Haskell code which will be executed sequentially.
This is also similar to our skeleton library \SkelCL, but the use of Haskell makes Accelerate hard to use for programmers coming from more traditional imperative and object-oriented languages like C and \Cpp.
Furthermore, \SkelCL has specific features, like the support of additional arguments, which enhance the flexibility in which it is used, allowing to efficiently implement real-world examples, \eg, the LM OSEM application discussed in \autoref{chapter:skelcl-evaluation}.

Accelerate uses high-level optimizations~\cite{McDonellCKL13} for fusing multiple skeleton executions into a single one -- an optimization technique known as \emph{deforestation}~\cite{Wadler90} in the functional community.
This is different from \SkelCL which always generates an \OpenCL kernel for each skeleton.
Therefore, Accelerate is able to generate efficient code for benchmarks where \SkelCL performs badly, including \emph{asum} and the dot product.
As discussed in \autoref{ch:seventh} we intend to address this drawback of \SkelCL in the future by incorporating our novel compilation technique which is able to perform this type of optimization as well.
Furthermore, Accelerate is a project specialized for \GPUs and its implementation is not portable across architectures, as our compilation technique is.

\bigskip

Many projects following the tradition of \emph{OpenMP}~\cite{OpenMP} working with annotations of sequential code have been proposed for \GPU programming as well.
Directives are used by the programmer to annotate sequential code, usually loops, which can be executed in parallel.
A compiler supporting then reads the directives and generates parallel code automatically.

Early projects generating \GPU code include \emph{HMPP}~\cite{DolbeauBiBo2007}, the \emph{PGI Accelerator Compilers} (PGI has since been acquired by Nvidia), and \emph{hiCUDA}~\cite{HanA11}.
All these projects contributed into a new unified standard called \emph{OpenACC}~\cite{OpenACC}.

\emph{OmpSs}~\cite{ElangovanBP12,DuranABLMMP11} is a project using directives with a particular focus on task parallelism.
Sequential code can be declared as a task via annotations and dependencies between tasks are specified by the programmer as well.
A runtime system can exploit parallelism by executing independent tasks simultaneously.
Data-parallel computations can also be executed on \GPUs using \OpenCL~\cite{ElangovanBP12}.
OmpSs influenced the development of the latest standard of \emph{OpenMP} 4.0~\cite{OpenMP} which now also includes directives for offloading computations to \GPUs.

In OpenACC and OpenMP 4.0 directives for specifying the parallel computation in loops are provided as well as directives for explicitly specifying the data regions involved in the computation.
After each computation data is copied back to the \CPU and many \GPU features can currently not be exploited with this approach, \eg, the usage of local memory.
Instead \SkelCL as well as our code generation technique fully exploit all features provided by \GPUs which are crucial for achieving high performance.
All these approaches using directives promise minimal source code change for existing sequential code.
\SkelCL requires programmers to express their programs in a well structured way using algorithmic skeletons, which might force programmers to reimplement parts of their programs.
We conceder this actually a benefit of \SkelCL, as restructuring the code using skeletons will likely increase its maintainability in the future.

\bigskip

Many new programming languages have been proposed for \GPU programming as well.
Some existing programming languages have been extended to support \GPU programming directly from the language without the need of specific libraries, including IBM's X10 language~\cite{TakeuchiMaKaHoSuSuOn2011}.

\bigskip

\emph{HiDP}~\cite{ZhangM13} is a language for expressing hierarchical data-parallel programs.
Fundamental data-parallel building blocks like map are pre-defined in the language and used by programmers.
Patterns can be nested, where this is not possible in \SkelCL but fully supported in our novel compilation technique.

\bigskip

\emph{LiquidMetal}~\cite{HuangHBR08} is a research project by IBM to support the programming of heterogeneous systems comprised of different hardware in a single programming language called \emph{Lime}~\cite{AuerbachBCR10}.
Support for programming \GPU systems has recently been added~\cite{DubachCRBF12}.
Lime supports task-parallelism with the creation of tasks which can communicate with each other.
Data-parallelism is also supported with skeleton-like patterns, including map and reduce.

\bigskip

\emph{Single Assignment C (SAC)}~\cite{GrelckS06} is a functional programming language supporting \GPUs programming by compiling SAC programs to CUDA~\cite{GuoTS11}.
The compiler automatically detects loops which can be offloaded to the \GPU, generated the necessary CUDA code and performs the data management automatically.
SAC has a special loop (\code{with}-loops) which are guaranteed to be free of dependencies, similar to \SkelCL's map skeleton.

\bigskip

\emph{Copperhead}~\cite{CatanzaroGK11} is a \GPU language embedded in the Python programming language offering data-parallel primitives, including map, reduce, and scan.
Nesting of patterns is supported and nested parallelism is statically mapped to the \GPU thread hierarchy by the compiler and can be controlled via compiler switches.
Currently, no systematic mechanism exists for exploring different mapping strategies.
Our code generation approach allows to systematically explore different implementation strategies, including different mappings of nested data parallelism to the hardware.

\bigskip

\emph{NOVA}~\cite{CollinsGGLS14} is a recent functional \GPU programming language developed at Nvidia.
It is intended to be used as an intermediate language produces by a domain specific programming interface.
Programs are implemented in a Lisp-style notation where function application is written using prefix notation.
Built-in parallel operations include map, reduce, and scan.

\bigskip

Finally, \emph{Petabricks}~\cite{AnselCWOZEA09} allows the programmer to specify a range of valid implementations for a given problem.
The programmer thereby specifies an implementation space the Petabricks compiler and runtime explores using autotuning strategies.
The two main concepts are \emph{transforms} which are functions which can be transformed by the Petrabricks compiler and \emph{rules} which describe a possible way for performing the computation.
By specifying multiple alternative rules the compiler is free to choose which one to apply at a given point in the program.
Recent work has enabled the generation of \GPU code as well~\cite{PhothilimthanaARA13}.

Different to our code generation approach for tackling performance portability, Petabricks relies on static analysis for optimizing the generated code.
Furthermore, the responsibility for specifying algorithmic choices is on the application programmer implementing a specific algorithm, whereas in our approach the algorithmic choices are once for all defined and captured in the rewrite rules and then applied to the high-level code written by the programmer.
Therefore, in our approach the algorithmic choices can be seen as implicit, or hidden from the user, where in Petabricks they are explicit to the programmer.

\bigskip
Obviously all these programming languages take a very different approach as \SkelCL does by requiring programmers to learn a new language and reimplement their entire application in the new language.
Interfacing code written in one of these new languages with legacy code is often complicated, whereas it is straightforward and easy when using \SkelCL, because it is implemented as an ordinary \Cpp library.

\subsection{Related Domain Specific Approaches}
% For Stencil:
% Patus
% PARTANS

% For Image processing:
% Halide
% Impalla ...?

\subsection{Related Approaches using Rewrite Rules}
% Playing by the rules ...
% Sergei ...
Spiral~\cite{OfenbeckRSOP13}

\section{Conclusion}


